\documentclass[twoside=false,
           twocolumn=false,
           a4paper,DIV=15,
           10pt]{scrartcl}


% using UTF8 as encoding for all files
\usepackage[utf8]{inputenc}

% provides '\includegraphics'
\usepackage{graphicx}

% provides auto-convert eps to pdfs
% \usepackage{epstopdf}

% provides cells spanning multiple rows in tables
\usepackage{multirow}

% American math society, base + symbol package (mathbb and so forth..)
\usepackage{amsmath,amssymb}

% provides double-stroked numbers, for e.g. unit matrix symbol 1
\usepackage{bbm}

% provides support for automated unit typesetting (e.g. 200 GeV)
\usepackage{units}

% provides \toprule etc. for nicer tables
\usepackage{booktabs}

% provides side-figures with text flow around them
\usepackage{wrapfig}

% language support, last option is standard for document
\usepackage[ngerman,english]{babel}

% subimport support for using cascaded \input command (gnuplot epslatex)
\usepackage{import}

%\usepackage[normal]{caption}

% change the (figure,etc.) counter, used for introduction
\usepackage{chngcntr}


% -----------------------------------------------------------
% Set / select font
% -----------------------------------------------------------

% Set the font
\usepackage[T1]{fontenc}

% Latin modern, successor of computer modern with correct umlauts
\usepackage{lmodern}	    



% -----------------------------------------------------------
% Somewhat more complex hyperlink setup 
% -----------------------------------------------------------

% provides extended colors, used here only for links
\usepackage{xcolor}

% link setup as suggested HERE: 
% http://tex.stackexchange.com/questions/823/remove-ugly-borders-around-clickable-cross-references-and-hyperlinks
\usepackage[pdfa,pdfencoding=auto,pdfusetitle
]{hyperref}
\definecolor{dark-red}{rgb}{0.4,0.15,0.15}
\definecolor{dark-blue}{rgb}{0.15,0.15,0.4}
\definecolor{medium-blue}{rgb}{0,0,0.5}
\hypersetup{
    colorlinks, linkcolor={dark-red},
    citecolor={dark-blue}, urlcolor={medium-blue},
    linktocpage=true
}

%
\setcapindent{1.5em}

% -----------------------------------------------------------

\usepackage{listings}

\definecolor{mygreen}{rgb}{0,0.6,0}
\definecolor{mygray}{rgb}{0.5,0.5,0.5}
\definecolor{mymauve}{rgb}{0.58,0,0.82}


\lstset{ %
  basicstyle=
    {\ttfamily\footnotesize},     % the size of the fonts that are used for the code
%  breakatwhitespace=true,          % sets if automatic breaks should only happen at whitespace
%  breaklines=true,                 % sets automatic line breaking
  commentstyle=\color{mygreen},    % comment style
  frame=leftline,
  xleftmargin=0.3em,
  keepspaces=true,                 % keeps spaces in text, useful for keeping indentation of code (possibly needs columns=flexible)
  keywordstyle=\color{blue},       % keyword style
  language=C++,                    % the language of the code
  numbers=none,                    % where to put the line-numbers; possible values are (none, left, right)
  showspaces=false,                % show spaces everywhere adding particular underscores; it overrides 'showstringspaces'
  showstringspaces=true,           % underline spaces within strings only
  showtabs=false,                  % show tabs within strings adding particular underscores
  stringstyle=\color{mymauve},     % string literal style
  tabsize=4,                       % sets default tabsize to 2 spaces
%  title=\lstname,                   % show the filename of files included with \lstinputlisting; also try caption instead of title
  morekeywords=[8]{Spinor,PsiEntry,SU3,Gaugefield,Lattice,LSite,LSiteIter,CommunicationBase}
}

% shorthand to listings inline work
\newcommand\cpp[1]{\lstinline{#1}}

% provides the cref command, automaticly puts "figure, table" etc. in refs
\usepackage{cleveref}


% ============================================================================
% Set a thick line for overfull boxes
% ----------------------------------------------------------------------------

\overfullrule=10pt


% ============================================================================
% list of new commands and shorthands
% ----------------------------------------------------------------------------

% The unit one 
\newcommand{\unitone}{\mathbbm{1}}

% The trace symbol
\DeclareMathOperator{\trace}{Tr}

% The "of order" O
\DeclareMathOperator{\ofOrder}{O}

% point-spread and optical transfer functions
\DeclareMathOperator{\PSF}{PSF}
\DeclareMathOperator{\OTF}{OTF}


% A TODO marker, easily found in the document printout and source code
\newcommand{\TODO}[1]{\textcolor{red}{[TODO]}\textcolor{blue}{(#1)}}
% A comment marker
\newcommand{\cmnt}[1]{\textcolor{orange}{[?: #1]}}

% see http://tex.stackexchange.com/questions/77816/hyperref-not-jumping-to-the-appropriate-location
% sees to that the links point to the top of a figure, not its caption
% do this here so it does not influence the epslatex plots
% done by gnuplot
\usepackage[all]{hypcap}




\usepackage[sort,numbers]{natbib}
\usepackage{bibentry}

\newcommand{\cbibentry}[1]{\textcolor{blue}{\bibentry{#1}}}
%\newcommand*{\doi}[1]{\href{http://dx.doi.org/\detokenize{#1}{\raggedright DOI: \detokenize{#1}}}}

\nobibliography*
% ========================================================================================
% document defintions

\author{\small{Marcel Müller}}
\title{Overview on SR-SIM publications}
\subtitle{A commented collection of SR-SIM publications,\\
focussed on algorithms and some engineering.} 
\date{\small \today}

% ========================================================================================
% The document itself

\begin{document}

\maketitle

\abstract{
This document contains a commented collection
of publications with a focus on super-resolution structured
illumination microscopy. This is mostly for me to keep
track on where to find certain aspects written down
and published, but might also be useful to others.

The publications listed here will be on the technical side
of things (construction of SR-SIM microscopes and reconstruction
algorithms), looking at applications is (a) too much and (b)
not really my field.
}

\tableofcontents
\newpage

\section{Fundamental papers}

The \textit{standard} SR-SIM microscope and reconstruction algorithm,
in use by all commercial systems and various home-built setups, 
is often attributed to and referred to as \textbf{Gustaffson-Heintzmann}-SIM:

\begin{itemize}
\item Conference proceeding by R. Heintzmann (1999):\\
\cbibentry{heintzmann1999laterally}
\item 2D SIM by M. Gustaffson (2000):\\
\cbibentry{gustafsson2000}
\item 3D SIM by M. Gustaffson (2008):\\ 
\cbibentry{gustafsson2008}
\item There is also this PhD thesis and the related paper (from 2000):\\
\cbibentry{frohn2000super} \\
\cbibentry{frohn2000true}
\end{itemize}

More precisely, Gustaffson-Heintzmann-SIM typically refers to
sinusoidal SIM illumination pattern, which are amenable to a
direct reconstruction approach. 
% (see \cref{sec-novel} for more detail).

\section{Building microscopes}

There are a few papers with details on how groups created home-built
SIM setups.

\begin{itemize}
\item Setup based on a Hamamatsu SLM (thus, non-binary phase control), 
2-beam SIM, 4 pattern orientations ($0, 45, 90, 135$, easy pattern generation). I think
this might be first SLM-based SIM paper (2009):\\
\cbibentry{chang2009isotropic}
\item Setup based on a TI DMD (thus binary), and using incoherent (LED) light:\\
\cbibentry{dan2013dmd}
\item Setup based on the Forth-DD ferro-electric SLM (binary, phase shift), device
also in use by Betzig, setup very fast:\\
\cbibentry{fastsim}
\end{itemize}

Also interesting: This paper has calculations (ray-tracing) on how
the polarization of the interfering light influences pattern contrast:\\
\cbibentry{o2012polarization}

\section{Reconstruction / Algorithms}

SIM reconstruction is usually a two-step process:
\begin{itemize}
\item Parameter estimation: Obtaining pattern orientation and frequency,
obtaining (global and something pattern-individual) phase offsets
\item Reconstruction: Band separation, shift and recombination through filters
\end{itemize}
The second part, the actual reconstruction, is rather straight-forward
to implement and can be obtained e.g. directly from the formulas in \cite{gustafsson2008}.
The first part, however, can in principle be done through different algorithms,
with varying performance and sometimes hard-to-find documentation.

\subsection{Parameter estimation}

\begin{itemize}
\item The SIM pattern causes a peak in the Fourier spectrum of a raw data frame under 
structured illumination. That peak can in principle be used to obtain pattern orientation,
frequency (position of the peak) and phase of the patter (complex phase of the peak):\\
\cbibentry{shroff2009-phase}\\
From the analysis in the paper and my experience,
this method will work o.k. if the resolution enhancement is not too high, i.e. if the
peak associated with the illumination pattern is not dampened too much by the OTF. As
an advantage, it is easy to implement.

\item \textbf{Approach by cross-correlation of separated bands}. The idea here is that the separated
spectra have overlapping regions, so the correct shift (angle, 
frequency, global phase, modulation depth) can
be found by maximizing the cross-correlation of these bands in respect to a complex shift
vector. The idea is already explained in one paragraph in \cite{gustafsson2008}, but not written
down in detail. A quite detailed description can however be found in this review:\\
\cbibentry{yang2015method}

To my knowledge, this is the \textbf{method of choice} to obtain SIM reconstruction parameters,
in use e.g. even for the current work of non-linear SIM. It is also the only method I know to
obtain pattern frequency and angle, while the phases can be refined by further means.

\item Iterative phase optimization: The cross-correlation will only yield one global phase,
with phase differences between patterns assumed as fixed (and set in the band separation matrix).
The iterative approach optimizes these phases by analyzing their shift through cross-correlation in
the raw data:\\
\cbibentry{wicker2013-phase1}

It seems like a very sound approach, but it takes some time to implement correctly.

\item Non-Iterative phase optimization: 
A follow-up to the last paper, this performs phase optimization in a single step. The algorithm
is easy to understand and implement, the paper provides comparisons to the iterative method
(performance similar for realistic SNRs):\\
\cbibentry{wicker2013-phase2}

\end{itemize}

\subsection{Filters}

The last step of a SIM reconstruction is the recombination of frequency bands. Since
the bands are OTF-corrected, a suitable filter needs to be applied. Typically, this is
a modified/generalized Wiener-type filter.

\begin{itemize}
\item Optimization of modulation depth, spatially varying illumination intensity and such.
A very early paper (2004), I'll also have to look into how well we could integrate some
of these ideas into our software:\\
\cbibentry{schaefer2004structured}

\item For good cameras, photon counts are dominated by Poisson noise (photon count statistics)
instead of Gaussian noise (electron read-out noise). Since the Wiener filter is not tuned to that,
other filter approaches might yield better results.\\
\cbibentry{chu2014image}
\end{itemize}

\subsection{2D vs. 3D: Optical sectioning}

SIM reconstructions can be done in either 2D (single
slice, one focal plane) or 3D (requires z-stack and 3D OTF).
Single slice reconstructions are independent of the
illumination mode (2-beam or 3-beam interference), and
even profit of 3-beam illumination for the following trick:

To reduce the out-of-focus light, i.e. to introduce optical
sectioning, the 2D OTF is re-weighted in such a way that
SIM bands do not contribute around their missing cone. To use this
trick, one either needs three-beam data (which has the medium band
overlapping all missing cones) or two-beam data set to less
than the maximum resolution improvement. The idea is mentioned
in the appendix of \cite{wicker2013-phase1}, and documented in
two more papers:\\
\cbibentry{o2014optimized}\\
and\\
\cbibentry{shaw2015high}

\section{Software}

Software implementations of the SIM reconstruction algorithm that I am
aware of:

\begin{itemize}
\item Of course \textbf{fairSIM}. Currently single-slice, with cross-correlation
parameter estimation, handles two-beam and three-beam data (more bands work, but
not through GUI), and offers optical sectioning.\\
\cbibentry{mueller2016open}
\item Also, there is OpenSIM, a collection of Matlab functions to build
and test SIM reconstructions:\\
\cbibentry{lal2016structured}
\item And another, bigger Matlab-based software, 
where however "standard" SIM seems not
to be the main focus:
\cbibentry{kvrivzek2016simtoolbox}
\end{itemize}

Also important, the "SIMcheck" plugin. No reconstruction, but thorough
analysis of the input data quality. Last time I've checked, mainly
for 3D SIM:\\
\cbibentry{ball2015simcheck}

As a side-note, the publications to cite when using
ImageJ and Fiji:

\begin{itemize}
\item \cbibentry{schneider2012nih}
\item \cbibentry{schindelin2012fiji}
\end{itemize}

And even more of a side-note, the localization software packages I sometimes
want to cite:

\begin{itemize}
\item QuickPALM:\\ \cbibentry{henriques2010quickpalm}
\item rapidSTORM:\\ \cbibentry{wolter2012rapidstorm}
\item ThunderSTORM:\\ \cbibentry{ovesny2014thunderstorm}
\item Comparison of localization microscopy software packages:\\
\cbibentry{sage2015quantitative}
\end{itemize}


\section{Non-linear SIM}

This is probably not a complete list, but the milestone papers
on extending SIM with "non-linear techniques" towards more than
the $2\times$ resolution enhancement:

\begin{itemize}
\item
Arguably among the first papers that promoted the idea:\\
\cbibentry{heintzmann2002saturated}
\item Work on non-linear SIM by Gustaffson
Gustafsson \\
\cbibentry{gustafsson2005nonlinear}\\
and\\
\cbibentry{rego2012nonlinear}
\item Betzig's 2015 Science paper, SIM-everything:
\cbibentry{li2015extended} 

\end{itemize}



\bibliography{literature}
\bibliographystyle{unsrturl}

\end{document}

